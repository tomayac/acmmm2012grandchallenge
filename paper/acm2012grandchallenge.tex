\documentclass{acm_proc_article-sp}

\usepackage[utf8]{inputenc}

\usepackage[T1]{fontenc}

\usepackage[activate=compatibility]{microtype}

% autoref command
\usepackage[pdftex,urlcolor=black,colorlinks=true,linkcolor=black,citecolor=black]{hyperref}
\def\sectionautorefname{Section}
\def\subsectionautorefname{Subsection}
\def\subfloatautorefname{Subfigure}

\usepackage[lofdepth,lotdepth]{subfig}

\usepackage{enumitem}

\usepackage{textcomp}

\usepackage{mathtools}

% give emph a normal fontsize
\let\oldemph\emph
\renewcommand{\emph}[1]{\oldemph{\fontsize{9}{9}\selectfont #1}}

% more readable footnote layout
\renewcommand{\footnotesize}{\fontsize{8pt}{10pt}}
\setlength{\footnotesep}{.5cm}

% todo macro
\usepackage{color}
\newcommand{\todo}[1]{\noindent\textcolor{red}{{\bf \{TODO}: #1{\bf \}}}}

% listings and Verbatim environment
\usepackage{fancyvrb}
\usepackage{relsize}
\usepackage{listings}
\usepackage{verbatim}
\newcommand{\defaultlistingsize}{\fontsize{8pt}{9.5pt}}
\newcommand{\inlinelistingsize}{\fontsize{8pt}{11pt}}
\newcommand{\smalllistingsize}{\fontsize{7.5pt}{9.5pt}}
\newcommand{\listingsize}{\defaultlistingsize}
\RecustomVerbatimCommand{\Verb}{Verb}{fontsize=\inlinelistingsize}
\RecustomVerbatimEnvironment{Verbatim}{Verbatim}{fontsize=\defaultlistingsize}
\lstset{frame=lines,captionpos=b,numberbychapter=false,escapechar=§,
        aboveskip=0.5em,belowskip=0em,abovecaptionskip=0em,belowcaptionskip=0em,
framexbottommargin=-1em,
        basicstyle=\ttfamily\listingsize\selectfont}

% use Courier from this point onward, except for URLs
\let\oldttdefault\ttdefault
\renewcommand{\ttdefault}{pcr}
\let\oldurl\url
\renewcommand{\UrlFont}{\fontfamily{\oldttdefault}\selectfont}

% linewrap symbol
\definecolor{grey}{RGB}{130,130,130}
\newcommand{\linewrap}{\raisebox{-.6ex}{\textcolor{grey}{$\hookleftarrow$}}}

% for 1st, 2nd, etc. superscripting
\newcommand{\ts}{\textsuperscript}

% more pleasing quote environment
\usepackage{tikz}
\newcommand*{\openquote}{\tikz[remember picture,overlay,xshift=-7pt,yshift=1pt]
     \node (OQ) {\fontfamily{fxl}\fontsize{16}{16}\selectfont``};\kern0pt}
\newcommand*{\closequote}{\tikz[remember picture,overlay,xshift=2pt,yshift=-4.5pt]
     \node (CQ) {\fontfamily{fxl}\fontsize{16}{16}\selectfont''};}
\renewenvironment{quote}%
{\setlength{\parindent}{1cm}\par\openquote}
{\closequote\vspace{-4.5pt}
}

\begin{document}

\title{NiteOutMag{\Large \textbf{\textsuperscript{\texttrademark}}}---At Least the Web\\ Remembers What Happened Last Night}

\numberofauthors{2}
\author{
% Tom, Ruben, Raphaël, Giuseppe, José
}
\maketitle

\begin{abstract}
The chorus of the popular song \emph{Tik Tok} by the artist \emph{Ke\$ha} goes
\emph{``Don't stop, make it pop. DJ, blow my speakers up. Tonight, I'mma fight.
'Til we see the sunlight. Tik tok on the clock. But the party don't stop, no.''}
We all know, however, that each party comes to an end eventually.
With NiteOutMag\texttrademark, we present a~Web application
that can help people recover what (the hell) happened last night.
Among the younger generation, nightlife activities---just about like any other
activity---together with related multimedia data get shared online on social networks.
The problem is that for one and the same event, the event-related user-generated data
may be shared on a~plethora of social networks.
Therefore, for this challenge, we introduce an application
that leverages event data from several event databases,
social data from multiple social networks, and media data from some media platforms.
The collected data for events around a~given area is then compiled
in an event-centric magazine-like way, where each page represents one event.
Just like the nightlives of people happen \emph{ad hoc} from one party to the other,
the NiteOutMag\texttrademark~application as well
gathers all its data on-the-fly, without the burden of pre-compiled data directories.
The application is available online\footnote{\todo{Add demo link}}.
A~screencast showcasing the application is also available\footnote{\todo{Add screencast link}}.
\end{abstract}

%\category{H.3.4}{Information Systems}{Information Storage and Retrieval}[World Wide Web]
%\category{H.3.5}{Online Information Services}{Web-based services}

%\keywords{}


\section{Introduction}
In March 2012, the at the time 901 million monthly active users
of the social networking site (SNS) Facebook
have uploaded more than 300 million photos on average per day~\cite{Facebook2012}.
Many of those photos are event-related, \emph{e.g.},
illustrate events like music concerts that a~social network user has attended.
Facebook, however, albeit the biggest, is only one among a~plethora of social networks
that event attendants may have shared their event-related content on.
In the next subsection, we thus list the SNSs considered for
NiteOutMag\texttrademark.

\subsection{Social Networks}
A~social network is an online service or media platform that focuses on building
and reflecting social relationships among people sharing interests and/or activities.
In this paper, we consider 11 different social networks
that represent all together most of the Western world's market share.
In detail, the considered social networks are \mbox{Google+} (\url{google.com/+}),
Myspace (\url{myspace.com}),
Facebook (\url{facebook.com}),
Twitter ({twitter.com}),
Instagram (\url{instagram.com}),
Flickr (\url{flickr.com}),
YouTube (\url{youtube.com}),
yfrog (\url{yfrog.com}),
MobyPicture (\url{mobypicture.com}),
Twitpic (\url{twitpic.com}), and
\mbox{img.ly} (\url{img.ly}).


\subsection{Event Databases}

\section{Related Work}
Illustrating events with media items stemming from social networks
is an active topic of research since the rise of social networking.
In~\cite{Brenner2012}, Brenner and Izquierdo present
an approach to detect social events and retrieve associated photos
in collaboratively annotated photo collections combining data of various modalities
such as time, location, and textual and visual features.
In~\cite{Liu2011}, Liu \emph{et al.} present a method
combining semantic inferencing and visual analysis for automatically gathering
media (photos and videos) illustrating events.

Regarding esthetic aspects of media compilation,
in~\cite{Sandhaus2011}, Sandhaus \emph{et al.} consider visual and
esthetic features for the automatic creation of photo books.
Obrador \emph{et al.}\ use visual and esthetic features
for a category-based approach to automatically assess
the esthetic appeal of photographs~\cite{Obrador2012}.


\section{Implementation}

\subsection{Geocoding---From Address to Lat/Long}

\subsection{Event Search---Things to Do at Lat/Long}

\subsection{Media Search---Photos Taken at an Event}

\subsection{Magazine Layout---Arranging Esthetically}

\subsection{Installation Instructions}

\section{Experiments}

\subsection{Query Examples}

\subsection{Discussion}

\section{Future Work and Conclusion}

% \section{Acknowledgments} 
% Double-blind review process, need to remove for now


\bibliographystyle{abbrv}
\bibliography{acm2012grandchallenge}

\balancecolumns
\end{document}